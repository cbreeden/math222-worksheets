%%%%%%%%%%%%%%%%%%%%%%%%%%%%%%%%%%%%%%%%%
% Short Sectioned Assignment
% LaTeX Template
% Version 1.0 (5/5/12)
%
% This template has been downloaded from:
% http://www.LaTeXTemplates.com
%
% Original author:
% Frits Wenneker (http://www.howtotex.com)
%
% License:
% CC BY-NC-SA 3.0 (http://creativecommons.org/licenses/by-nc-sa/3.0/)
%
%%%%%%%%%%%%%%%%%%%%%%%%%%%%%%%%%%%%%%%%%

%----------------------------------------------------------------------------------------
%	PACKAGES AND OTHER DOCUMENT CONFIGURATIONS
%----------------------------------------------------------------------------------------

\documentclass[letterpaper, fontsize=11pt]{scrartcl} % A4 paper and 11pt font size
\usepackage{geometry} 
\geometry{left=0.8in, right =0.8in, bottom=1in, top=0.6in} % Establish a 1 inch margin

\usepackage{graphicx} % Allow ability to include graphics

\usepackage[T1]{fontenc} % Use 8-bit encoding that has 256 glyphs
\usepackage{fourier} % Use the Adobe Utopia font for the document
\usepackage[english]{babel} % English language/hyphenation
\usepackage{amsmath,amsfonts,amsthm} % Math packages

\usepackage{enumitem}
\usepackage{tabularx}

\usepackage{sectsty} % Allows customizing section commands
\allsectionsfont{\centering \normalfont\scshape\underline} % Make all sections centered, the default font and small caps

\usepackage{fancyhdr} % Custom headers and footers
\pagestyle{fancyplain} % Makes all pages in the document conform to the custom headers and footers
\fancyhead{} % No page header - if you want one, create it in the same way as the footers below
\fancyfoot[L]{} % Empty left footer
\fancyfoot[C]{} % Empty center footer
\fancyfoot[R]{\thepage} % Page numbering for right footer
\renewcommand{\headrulewidth}{0pt} % Remove header underlines
\renewcommand{\footrulewidth}{0pt} % Remove footer underlines
\setlength{\headheight}{13.6pt} % Customize the height of the header

\numberwithin{equation}{section} % Number equations within sections (i.e. 1.1, 1.2, 2.1, 2.2)
\numberwithin{figure}{section} % Number figures within sections (i.e. 1.1, 1.2, 2.1, 2.2)
\numberwithin{table}{section} % Number tables within sections (i.e. 1.1, 1.2, 2.1, 2.2)

\setlength\parindent{0pt} % Removes all indentation from paragraphs - comment this line for an assignment with lots of text

%----------------------------------------------------------------------------------------
%	TITLE SECTION
%----------------------------------------------------------------------------------------
\title{Math 222 - Worksheet 5}
\newcommand{\horrule}[1]{\rule{\linewidth}{#1}} % Create horizontal rule command with 1 argument of height

\begin{document}

%% Header
{
\normalfont \normalsize 
\begin{flushright}
\textsc{Math 222, UW Madison}
\end{flushright}
{\center
\horrule{0.5pt} \\[0.4cm] % Thin top horizontal rule
{\huge Worksheet 5}\\
Partial Fractions \\ % The assignment title
\horrule{2pt} \\[0.5cm] % Thick bottom horizontal rule
}}


%----------------------------------------------------------------------------------------
%	PROBLEM 1
%----------------------------------------------------------------------------------------

Whenever we want to integrate a rational function (that is one of the form $\displaystyle \frac{P(x)}{Q(x)}$ where $P(x)$ and $Q(x)$ are polynomials), we need to make sure that the polynomial in the numerator is {\bfseries smaller} than the one in the denominator.  We do this with {\bfseries polynomial division}.

\section*{Using Polynomial Division}

1. \quad $\displaystyle \int \frac{x^3 + x^2 + x + 2}{x^2 + 1}$

\vfill

\section*{Canonical Examples}
When working with rational functions you will run into three canonical examples, two of which require you to brush up on your {\bfseries completing the square} skills.  The canonical examples are of the form:

\[ \int \frac{1}{(x-a)^2 + b^2}\,dx, \qquad \int \frac{1}{(x-a)^n}\,dx, \qquad \int \frac{x}{(x - a)^2 + b^2}\,dx \]

There are a few more slightly more complicated cases, but knowing how to handle these 3 are absolutely critical.  Just keep in mind when you see these guys you {\bfseries don't} want to use partial fractions.\\

2. \quad $\displaystyle \int \frac{1}{x^2  - 2x + 1}\,dx$
\vfill

\newpage

3. \quad $\displaystyle \int \frac{1}{x^2 -2x + 1}\,dx$

\vfill

4. \quad $\displaystyle \int \frac{1}{2x^2 + 4x + 10}\,dx$

\vfill

5. \quad $\displaystyle \int \frac{x}{x^2 + 2x + 2}\,dx$

\vfill

\newpage

6. \quad $\displaystyle \int \frac{x + 1}{x^2 - 2x + 1}\,dx$

\vfill

\section*{Partial Fraction Decomposition}

Remember these general steps for rational functions:

\begin{enumerate}
\item Perform polynomial division, if necessary
\item Factor the denominator (the quadratic formula may be helpful)
\item Perform partial fractions decomposition [PFD] (only if you have multiple factors)
\item Solve for unknowns in PFD using either Heaviside or solving a system of equations.
\end{enumerate}

6. \quad $\displaystyle \int \frac{x^5}{x^4 - 1}\,dx$

\vfill

\newpage

7. \quad $\displaystyle \int_0^1 \frac{x^4(1-x)^4}{1 + x^2}\,dx$

\vfill

8. \quad $\displaystyle \int \frac{w - 1}{w^2 + w}\,dx$

\vfill

\newpage

9. \quad $\displaystyle \int \frac{x}{x^4 - 2x^2 + 1}\,dx$

\vfill

10. \quad $\displaystyle \int \frac{x}{x^3 + 2x^2 + 2x}\,dx$

\vfill



\end{document}




