%%%%%%%%%%%%%%%%%%%%%%%%%%%%%%%%%%%%%%%%%
% Short Sectioned Assignment
% LaTeX Template
% Version 1.0 (5/5/12)
%
% This template has been downloaded from:
% http://www.LaTeXTemplates.com
%
% Original author:
% Frits Wenneker (http://www.howtotex.com)
%
% License:
% CC BY-NC-SA 3.0 (http://creativecommons.org/licenses/by-nc-sa/3.0/)
%
%%%%%%%%%%%%%%%%%%%%%%%%%%%%%%%%%%%%%%%%%

%----------------------------------------------------------------------------------------
%	PACKAGES AND OTHER DOCUMENT CONFIGURATIONS
%----------------------------------------------------------------------------------------

\documentclass[paper=a4, fontsize=11pt]{scrartcl} % A4 paper and 11pt font size
\usepackage{geometry} 
\geometry{left=0.8in, right =0.8in, bottom=1in, top=0.6in} % Establish a 1 inch margin

\usepackage{graphicx} % Allow ability to include graphics

\usepackage[T1]{fontenc} % Use 8-bit encoding that has 256 glyphs
\usepackage{fourier} % Use the Adobe Utopia font for the document
\usepackage[english]{babel} % English language/hyphenation
\usepackage{amsmath,amsfonts,amsthm} % Math packages

\usepackage{enumitem}
\usepackage{tabularx}

\usepackage{sectsty} % Allows customizing section commands
\allsectionsfont{\centering \normalfont\scshape\underline} % Make all sections centered, the default font and small caps

\usepackage{fancyhdr} % Custom headers and footers
\pagestyle{fancyplain} % Makes all pages in the document conform to the custom headers and footers
\fancyhead{} % No page header - if you want one, create it in the same way as the footers below
\fancyfoot[L]{} % Empty left footer
\fancyfoot[C]{} % Empty center footer
\fancyfoot[R]{\thepage} % Page numbering for right footer
\renewcommand{\headrulewidth}{0pt} % Remove header underlines
\renewcommand{\footrulewidth}{0pt} % Remove footer underlines
\setlength{\headheight}{13.6pt} % Customize the height of the header

\numberwithin{equation}{section} % Number equations within sections (i.e. 1.1, 1.2, 2.1, 2.2)
\numberwithin{figure}{section} % Number figures within sections (i.e. 1.1, 1.2, 2.1, 2.2)
\numberwithin{table}{section} % Number tables within sections (i.e. 1.1, 1.2, 2.1, 2.2)

\setlength\parindent{0pt} % Removes all indentation from paragraphs - comment this line for an assignment with lots of text

%----------------------------------------------------------------------------------------
%	TITLE SECTION
%----------------------------------------------------------------------------------------
\title{Math 222 - Worksheet 1}
\newcommand{\horrule}[1]{\rule{\linewidth}{#1}} % Create horizontal rule command with 1 argument of height

\begin{document}

%% Header
{
\normalfont \normalsize 
\begin{flushright}
\textsc{Math 222, UW Madison}
\end{flushright}
{\center
\horrule{0.5pt} \\[0.4cm] % Thin top horizontal rule
{\huge Worksheet 1}\\
Review \\ % The assignment title
\horrule{2pt} \\[0.5cm] % Thick bottom horizontal rule
}}


%----------------------------------------------------------------------------------------
%	PROBLEM 1
%----------------------------------------------------------------------------------------

\section*{Trig Identities}

1. Using the identity $\sin^2(\theta) + \cos^2(\theta) = 1$, show that $\tan^2(\theta) + 1 = \sec^2(\theta)$.
\vfill

2. For each of the following, circle the correct answer:
\begin{align*}
2\sin(\theta)\cos(\theta) = && \cos(2\theta) && \sin(2\theta) \\
\cos^2(\theta) - \sin^2(\theta) = && \cos(2\theta) && \sin(2\theta) \\
\cos^2(\theta) = && \frac{1}{2}\big(1 + \cos(2\theta)\big) && \frac{1}{2}\big(1 - \sin(2\theta)\big) \\
\tan(\arctan(x)) = && 1 && x
\end{align*}\newline

\section*{Calculus}

3. True or False:\\

%\renewcommand{\arraystretch}{2}
\begin{tabularx}{\textwidth}{ X X }
a. \quad $\displaystyle \frac{d}{dx}\left(\frac{1}{x}\right) = \ln x$ &
d. \quad The function $\displaystyle f(x) = \frac{1}{x+4}$ is defined for all values $x$ except for $x = -4$. \\
\noalign{\smallskip}\noalign{\smallskip}\noalign{\smallskip}
b. \quad $\displaystyle \frac{d}{dt} \int_0^t \frac{1}{1+x^2}\,dx = \frac{1}{1+t^2}$ &
e. \quad $\displaystyle \int e^{x^2}\,dx = e^{x^2} + C$ \\
\noalign{\smallskip}\noalign{\smallskip}\noalign{\smallskip}\noalign{\smallskip}
c. \quad $\displaystyle \sqrt{x^2 + 9} = x + 3$ &
f. \quad If $f(x) = x^2 \cdot g(x)$ then $f'(x) = 2x \cdot g'(x)$.
\end{tabularx}\newline\newline\newline\newline

4. For each of the following, state whether the object is a function or a number.  If it is a function, state what variable it is a function of.\\

\begin{tabularx}{\textwidth}{ X X }
a. \quad $\displaystyle \int_1^x e^{\cos(t)}\,dt$ &
c. \quad $\displaystyle \int_1^3 \sin(s)\,ds$ \\
\noalign{\smallskip}\noalign{\smallskip}\noalign{\smallskip}\noalign{\smallskip}
b. \quad $\displaystyle \int_t^{t^2}\,\ln(\cos(x))\,dx$ &
d. \quad $\displaystyle \int \ln(x)\,dx$
\end{tabularx}

\newpage

5. Compute $\displaystyle \frac{d}{dx} \int_x^1 \ln z\,dz$.  (Hint: Recall the "Fundamental Theorem of Calculus")

\vfill

6. Compute $\displaystyle \int e^x\sin(2\pi e^x)\,dx$.

\vfill

7. Compute $\displaystyle \int_0^x \left( \int_0^t \cos(s)\,ds \right)\,dt$.

\vfill

8. Define $\displaystyle f(x) = \ln(x)\sin(x) + \sqrt{x^4 + x^2}$.  Compute $f'(x)$.

\vfill








\end{document}




