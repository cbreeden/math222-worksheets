\newproblem{euler:nonlinear}{%
Suppose that $p$ is a function of $t$, satisfying 
\begin{align*}
tp' &= a(p^2 + t^2) \\ 
p(1) &= 2
\end{align*}
where $a$ is some small unknown constant.  Using Euler's method with step size $2$, approximate $p(5)$.  Your answer will depend on $a$.}{%

We know $p(1)=2$.  Plugging into the equation, 
\[1*p'(1) = a(2^2 + 1^2)\]
so $p'(1) = 5a$.  Thus
\[p(3) \approx 2*p'(1) + p(1) = 2*5a + 2 = 10a + 2\]

We know $p(3) \approx 10a + 2$.  Plugging into the equation, 
\[3*p'(3) \approx a((10a+2)^2 + 9\]
So $p'(3) \approx \frac{a}{3}(10a+2)^2 + 3$.  Thus
\[p(5) \approx 2*p'(3) + p(3) = 2*\left(\frac{a}{3}(10a+2)^2 + 3 \right) + 10a + 2\]
}

\newproblem{euler:sin(0.3)}{%
Use Euler's method with step size $0.1$ to approximate $\sin(0.3)$}{%

To apply Euler's method we need a differential equation satisfied by sin, and an initial condition.  $y=\sin$ satisfies
\[y'' = -y\]
As for the initial condition, we don't know the value of $\sin(x)$ for most numbers $x$ so our choice of useful initial conditions is limited - for example, the initial condition $x=1$, $y=\sin(1)$ is useless to us since we don't know $\sin(1)$.  But we do know $y(0) = 0$, $y'(0) = 1$ so let's use that, starting at $x=0$.

We have $y(0) = 0$, $y'(0) = 1$.  Plugging into $y'' = -y$, we find that $y''(0) = -0 = 0$.  Since $y'(0) = 1$ we have 
\[y(0.1) \approx 0.1*y'(0) + y(0) \approx 0.1*1 + 0 = 0.1\]
and since $y''(0) \approx 0$ we have 
\[y'(0.1) \approx 0.1*y''(0) + y'(0) \approx 0.1*0 + 1 = 1\]

We have $y(0.1) \approx 0.1$, $y'(0.1) \approx 1$.  Plugging into $y'' = -y$, we find that $y''(0) \approx -0.1$.  Since $y'(0.1) \approx 1$ we have 
\[y(0.2) \approx 0.1*y'(0.1) + y(0.1) \approx 0.1*1 + 0.1 = 0.2\]
and since $y''(0) \approx -0.1$ we have 
\[y'(0.2) \approx 0.1*y''(0.1) + y'(0.1) \approx 0.1*-0.1 + 1 = 0.9\]

We have $y(0.2) \approx 0.2$, $y'(0.2) \approx 0.9$.  Sicne we only need to do one more step I won't bother finding $y''(0.2)$.  Instead,
\[y(0.3) \approx 0.1*y'(0.2) + y(0.2) \approx 0.1*0.9 + 0.2 = 0.29\]
and our final answer is 
\[\sin(0.3) = y(0.3) \approx 0.29\]

One drawback of this approach is that there is no way to estimate your error, but in fact this answer is correct to 2 decimal places.
}

\newproblem{eulersMethod:Gaussian}{%
Solve the following initial value problem exactly, then use Euler's method with step size $\Delta x =.1$ to estimate $y(.3)$.
\begin{align*}
\frac{dy}{dx} &= -2xy \\
y(0) = 1
\end{align*}}{%
This differential equation is separable and the solution is $y(x) = e^{-x^2}$.

To use Euler's method with step size $.1$, we will iteratively compute estimates to $y(.1), y(.2)$ and $y(.3)$.

First, we need an estimate for $y(.1) \approx y(0) + \frac{dy}{dx}(0)\Delta x$, so we need to compute $\frac{dy}{dx}(0)$.
\begin{align*}
\frac{dy}{dx}(0) = -2(0)y(0) = 0
\end{align*}

so we estimate that 
\begin{align*}
y(.1) &\approx y(0) + \frac{dy}{dx}(0)(.1) \\
&= 1 + (0)(.5) = 1
\end{align*}

Now we repeat the procedure above to find an approximation for $y(.2)$
\begin{align*}
y(.2) \approx y(.1) + \frac{dy}{dx}(.1) \Delta x
\end{align*}
 To compute $\frac{dy}{dx}(.1)$, we recall that we have the differential equation $\frac{dy}{dx} = -2xy$, so $\frac{dy}{dx}(.1) = -2 (.1) y(.5)$.  We have estimated that $y(.1) = 1$, so we use that in our computation, and get $\frac{dy}{dx}(.5) \approx -2(.1)(1) = -.2$.  This gives us the approximation
\begin{align*}
y(.2) &\approx y(.1) + \frac{dy}{dx}(.1) \Delta x \\
&\approx 1 + (-.2)(.1) = .98
\end{align*}

Finally, we compute an approximation to $y(.3) \approx y(.2) + \frac{dy}{dx}(.2) \Delta x$.  We first compute 
\begin{align*}
\frac{dy}{dx}(.2) &= -2 (.2) y(.2) \\
&\approx -2(.2)(.98) \\
&= -.392 
\end{align*}

Then our approximation to $y(.3)$ is
\begin{align*}
y(.3) &\approx y(.2) + \frac{dy}{dx}(.2) \Delta x \\
&\approx .98 + (-.392)(.1) \\
&= .9408
\end{align*}

As a comment, the true value of $e^{-(.3)^2}$ is about .914.
}

\newproblem{eulersMethod:linear1}{%
Solve the following initial value problem exactly, then use Euler's method with a step size of $\Delta x = .1$ to approximate $y(1.3)$.
\begin{align*}
\frac{dy}{dx} &= 1 + y \\
y(1) = 0
\end{align*}}{%
This equation is first order linear, so to solve it exactly, we rewrite it as
\begin{align*}
\frac{dy}{dx} - y &= 1
\end{align*}

The integrating factor is $m(x) = e^{-x}$, which converts this equation to
\begin{align*}
\underbrace{e^{-x}\frac{dy}{dx} - e^{-x} y}_\textrm{$\frac{d}{dx} (e^{-x}y)$} = e^{-x} \\
e^{-x}y &= -e^{-x} + C \\
y&= -1 + Ce^{x}
\end{align*}

Susbtituting in the initial condition gives the solution $y(x)  = -1 + e^{x-1}$.

We can now use Euler's method to approximate $y(.3)$.  We first compute
\begin{align*}
y(1.1) \approx y(1) + \frac{dy}{dx}(1) \Delta x
\end{align*}

From the differential equation $\frac{dy}{dx} = 1 + y$ and initial condition $y(1) = 0$ we find that $\frac{dy}{dx}(1) = 1 + y(1) = 1$. Therefore, we approximate
\begin{align*}
y(1.1) &\approx y(1) + \frac{dy}{dx}(1) \Delta x \\
&= 0 + (1)(.1) \\
&= .1
\end{align*}

Continuing as in the previous step,
\begin{align*}
y(1.2) &\approx y(1.1) + \frac{dy}{dx}(1.1) \Delta x
\end{align*}

where $\frac{dy}{dx}(1.1) = 1 + y(1.1) \approx 1 + .1 = 1.1$, so that
\begin{align*}
y(1.2) &\approx y(1.1) + \frac{dy}{dx}(1.1) \Delta x \\
&\approx .1 + (1.1)(.1) \\
&= .21
\end{align*}

Finally, we approximate
\begin{align*}
y(1.3) &\approx y(1.2) + \frac{dy}{dx}(1.2) \Delta x
\end{align*}

where $\frac{dy}{dx}(1.2) = 1 + y(1.2) \approx 1 + .21 = 1.21$ so that

\begin{align*}
y(1.3) &\approx y(1.2) + \frac{dy}{dx}(1.2) \Delta x \\
&\approx .21 + (1.21)(.1) \\
&= .331
\end{align*}
}

\newproblem{eulersMethod:sep1}{%
Find an exact solution to the following initial value problem, then use Euler's method with step size $\Delta x = .1$ to estimate $y(.2)$.
\begin{align*}
\frac{dy}{dx} &= -xy + 2x
\end{align*}}{%
}

\newproblem{eulersMethod:q1}{%
Find an exact solution to the following initial value problem, then 
use Euler's method with step size $\Delta x = .1$ to estimate $y(.2)$
\begin{align*}
\frac{dy}{dx} &= 2xy + x \\
y(0) &= 0
\end{align*}}{%
We begin by putting the differential equation into standard form
\begin{align*}
\frac{dy}{dx} - 2xy &= x
\end{align*}
The integrating factor for this problem is $m(x) = e^{\int -2xdx} = e^{-x^2}$.  Multiplication turns this into
\begin{align*}
\underbrace{e^{-x^2}\frac{dy}{dx} - 2xe^{-x^2}y}_\textrm{$\frac{d}{dx} e^{-x^2}y$} &= x e^{-x^2} \\
e^{-x^2}y &= \int xe^{-x^2} dx \\
&= \frac{-1}{2}e^{-x^2} + C \\
y(x) &= -\frac{1}{2} + Ce^{x^2}
\end{align*}
Substituting in the initial condition $y(0) = 0$ gives that $y(x) = \frac{1}{2}e^{x^2} - \frac{1}{2}$.

To approximate $y(.2)$ we first need an approximation for $y(.1)$.
\begin{align*}
y(.1) &\approx y(0) + \frac{dy}{dx}(0) \Delta x
\end{align*}

where $\frac{dy}{dx}(0) = 2(0)y(0) + 0 = 0$.  So we have $\frac{dy}{dx}(0) \approx y(0) + 0 (.1) = 0$.  We now have

\begin{align*}
y(.2) &\approx y(.1) + \frac{dy}{dx}(.1) \Delta x \\
\end{align*}

where $\frac{dy}{dx}(.1) = 2(.1)y(.1) + .1 = 2(.1)(0) + .1 = .1$.  Then, we have
\begin{align*}
y(.2) &\approx y(.1) + \frac{dy}{dx}(.1) \Delta x \\
&\approx 0 + .1(.1) \\
&= .01
\end{align*}
}

\newproblem{eulersMethod:q2}{%
Find an explicit solution to the following initial value problem, then use Euler's method with step size $\Delta x = .1$ to estimate $y(.2)$.
\begin{align*}
\frac{dy}{dx} &= 2xy + 2x \\
y(0) &= 0 
\end{align*}
}{%
We begin by observing that this equation is separable
\begin{align*}
\frac{dy}{dx} &= 2xy + 2x \\
&= 2x(1+y) \\
\int \frac{dy}{1+y} &= \int 2x dx \\
\ln|1 + y| &= x^2 + C \\
1 + y &= \underbrace{\pm e^C}_\textrm{$B \neq 0$} e^{x^2} \\
y &= B e^{x^2} - 1
\end{align*}

Substituting in the initial condition $y(0) = 0$ we find that $B = 1$ and $y(x) = e^{x^2} - 1$.

For Euler's method, we first estimate $y(.1)$
\begin{align*}
y(.1) \approx y(0) + \frac{dy}{dx}(0) \Delta x
\end{align*}
where $\frac{dy}{dx}(0) = 2(0)y(0) + 2(0) = 0$.  Then our approximation is
\begin{align*}
y(.1) \approx 0 + 0(.1) = 0
\end{align*}

Now we approximate $y(.2)$.
\begin{align*}
y(.2) &\approx y(.1) \frac{dy}{dx}(.1) \Delta x
\end{align*}
where $\frac{dy}{dx}(.1) = 2(.1)y(.1) + 2(.1) \approx 2(.1)(0) + 2(.1) = .2$.  Then, we have the approximation
\begin{align*}
y(.2) &\approx y(.1) + \frac{dy}{dx}(.1) \Delta x \\
&\approx 0 + (.2)(.1) \\
&= .02
\end{align*}
}