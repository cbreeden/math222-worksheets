\newproblem{taylor:sinx2}{%
Compute the second order Taylor polynomial of $\sin(x^2)$ around $0$ and use this to approximate $\sin(\frac{1}{4})$.}{%
We need to compute two derivatives of $f(x) = \sin(x^2)$.
\begin{align*}
f(x) &= \sin(x^2) \\
f'(x) &= \cos(x^2)(2x) \\
f^{(2)}(x) &= 2\cos(x^2) -4x^2 \sin(x^2)
\end{align*}

Now we evaluate the function and its derivatives at zero.

\begin{align*}
f(0) &= \sin(0) = 0 \\
f'(0) &= \cos(0)(0) = 0 \\
f^{(2)}(0) &= 2\cos(0) - 4(0)\sin(0) = 2
\end{align*}

Therefore the degree two Taylor polynomial of $\sin(x^2)$ is $\frac{2}{2!}x^2 = x^2$.

Since $\sin\left(\frac{1}{4}\right) = \sin\left(\left(\frac{1}{2}\right)^2\right)$, our approximation is $\left(\frac{1}{2}\right)^2 = \frac{1}{4}$.
}

\newproblem{taylor:etan}{%
Compute the degree two Taylor polynomial of the function $f(x) = e^{\tan(x)}$ around $0$.  Use this to estimate $e^{\tan(.1)}$.}{%
\begin{align*}
f(x) &= e^{\tan(x)} \\
f'(x) &= \sec^2(x) e^{\tan(x)} \\
f^{(2)}(x) &= 2 \sec^2(x)\tan(x)e^{\tan(x)} + \sec^4(x)e^{\tan(x)}
\end{align*}

We evaluate these at zero.
\begin{align*}
f(0) &= e^{\tan(0)} = 1 \\
f'(0) &= \sec^2(0) e^{\tan(0)} = 1 \\
f^{(2)}(0) &= 2 \sec^2(0)\tan(0)e^{\tan(0)} + \sec^4(0) e^{\tan(0)} = 1
\end{align*}

This gives the Taylor polynomial $1 + \frac{1}{1!}x + \frac{1}{2!}x^2 = 1 + x + \frac{1}{2}x^2$.  We can then approximate $e^{\tan(.1)} \approx 1 + .1 + \frac{1}{2} (.1)^2 = 1.105$.
}

\newproblem{taylor:sinexp}{% 
Compute the second order Taylor polynomial of $\sin(e^{x}-1)$ around $0$ and use this to approximate $\sin(e^{\frac{1}{2}}-1)$.}{%
We first need to compute two derivatives of $\sin(e^{x}-1)$
\begin{align*}
f(x) &= \sin(e^{x}-1) \\
f'(x) &= \cos(e^{x}-1)e^{x} \\
f^{(2)}(x) &= \cos(e^{x}-1)e^{x} - \sin(e^{x} -1)e^{2x}
\end{align*}

We evaluate these at zero.
\begin{align*}
f(0) &= \sin(0) = 0\\
f'(0) &= \cos(0)e^0 = 1 \\
f^{(2)}(0) &= \cos(0)e^0 - \sin(0)e^0 = 1
\end{align*}

Combining these, we find that the second order Taylor polynomial of $\sin(e^{x} -1)$ is $x + \frac{1}{2}x^2$.  This gives the approximation $\sin(e^{\frac{1}{2}} -1) \approx \frac{1}{2} + \left(\frac{1}{2}\right)^3 = \frac{5}{8}$.
}

\newproblem{taylor:polynomial}{%
Find the second and fourth order Taylor expansions around $1$ for the function $f(x) = x^3 + 5x + 1$.}{%
We can first observe that since this function is a third order polynomial, the fourth order Taylor expansion of $f(x)$ is $f(x)$.  To find the second order Taylor expansion, we need to differentiate:

\begin{align*}
f(x) &= x^3 + 5x + 1 \\
f'(x) &= 3x^2 + 5 \\
f^{(2)}(x) &= 6x
\end{align*}

Now we evaluate at zero.

\begin{align*}
f(1) &= 7 \\
f'(1) &= 8 \\
f^{(2)}(1) &= 6 \\
\end{align*}

So the second order Taylor polynomial around $1$ is $7 + 8(x-1) + \frac{6}{2}(x-1)^2 = 7 + 8(x-1) + 3(x-1)^2$.
}

\newproblem{taylor:intexp2}{%
Find the second order Taylor polynomial around $0$ for $f(x) = \int_0^x e^{-t^2}dt$ and use this to estimate $f(.1)$.}{%

\begin{align*}
f(x) &= \int_0^x e^{-t^2}dt \\
f'(x) &= e^{-x^2} \\
f^{(2)}(x) &= -2x e^{-x^2}
\end{align*}

so that 

\begin{align*}
f(0) &= \int_0^0 e^{-t^2}dt = 0 \\
f'(0) &= e^{-0^2} = 1 \\
f^{(2)}(0) &= -2(0)e^{-0^2} = 0
\end{align*}

so that the degree $2$ Taylor polynomial for $f(x)$ is $x$.  Our estimate for $f(.1)$ is therefore $.1$.
}

\newproblem{taylor:intexpsin}{%
Find the first order Taylor polynomial for the function $f(x) = \int_0^{\sin(x)} e^{-t^3} dt$ and use this to find an approximation for $f(\frac{1}{2})$.}{%
\begin{align*}
f(x) &= \int_0^{\sin(x)} e^{-t^3} dt \\
f'(x) &= e^{-\sin^3(x)}\cos(x) 
\end{align*}

so that 
\begin{align*}
f(0) &= \int_0^{\sin(0)} e^{-t^3} dt =\int_0^0 e^{-t^3} dt = 0 \\
f'(0) &= e^{-\sin^3(0)}\cos(0) = 1
\end{align*}

so the first order Taylor polynomial for $f(x)$ is given by $x$ and our approximation for $f(\frac{1}{2})$ is $\frac{1}{2}$.
}

\newproblem{taylor:intcomp}{%
Find the second order Taylor polynomial of $\cos(x)$ around $0$ then integrate this polynomial.  Additionally, find the third order Taylor polynomial of $\sin(x)$ around $0$ .  Recall that $\int \cos(x)dx = \sin(x) + C$ and compare your answer to the previously computed Taylor polynomial for the integral of $\cos(x)$.}{%
We begin by calling $f(x) = \cos(x)$ and $g(x) = \sin(x)$.  We need to compute two derivatives of $f$ and three derivatives of $g$.
\begin{align*}
f(x) &= \cos(x) \\
f'(x) &= -\sin(x) \\
f^{(2)}(x) &= -\cos(x) \\
g(x) &= \sin(x) \\
g'(x) &= \cos(x) \\
g^{(2)}(x) &= -\sin(x) \\
g^{(3)}(x) &= -\cos(x)
\end{align*}

so that
\begin{align*}
f(0) &= 1 \\
f'(0) &= 0 \\
f^{(2)}(0) &= -1 \\
g(0) &= 0 \\
g'(0) &= 1 \\
g^{(2)}(0) &= 0 \\
g^{(3)}(0) &= -1
\end{align*}

The degree two Taylor polynomial of $\cos(x)$ around $0$ is then $1 - x^2$ and the integral of this is $C + x - \frac{x^3}{3}$.  Similarly, we find that the degree three Taylor polynomial of $\sin(x)$ is $x - \frac{x^3}{3}$.  For $C = 0$, these agree--for nice functions, we can exchange the operation of taking Taylor polynomials and integration.
}

\newproblem{taylor:arctanseries}{%
Find the Taylor series around $0$ for $\mbox{arctan}(x)$, $T_\infty^0 \mbox{arctan}(x)$.}{%
\begin{align*}
\mbox{arctan}(x) &= \int \underbrace{\frac{1}{1+x^2}}_\textrm{$\frac{1}{1-(-x^2)}$} dx \\
&= \int \sum_{n=0}^\infty (-x^2)^n dx \\
&= \int \sum_{n=0}^\infty (-1)^n x^{2n} dx \\
&= \sum_{n=0}^\infty (-1)^n \int x^{2n} dx \\
&= C + \sum_{n=0}^\infty \frac{(-1)^n}{2n+1} x^{2n+1} \\ 
\end{align*}

Now we need to solve for $C$, which we can do by observing that $\mbox{arctan}(0) = 0$, so $C = 0$ and $\mbox{arctan}(x) = \sum_{n=0}^\infty \frac{(-1)^n}{2n+1} x^{2n+1}$.
}

\newproblem{taylor:cosh2series}{%
Find the Taylor series around zero for $\cosh(2x) = \frac{1}{2}\left(e^{2x} + e^{-2x}\right)$.}{%
\begin{align*}
\frac{1}{2}\left(e^{2x} + e^{-2x}\right) &= \frac{1}{2}\left(\sum_{n=0}^\infty\frac{(2x)^n}{n!} + \sum_{n=0}^\infty \frac{(-2x)^n}{n!}\right) \\
&= \frac{1}{2}\left(\sum_{n=0}^\infty\frac{2^nx^n}{n!} + \frac{(-1)^n 2^n x^n}{n!}\right) \\
&= \frac{1}{2}\sum_{n=0}^\infty \frac{1+(-1)^n}{n!}2^n x^n
\end{align*}
We can observe that $1 + (-1)^n = 0$ if $n$ is odd and $2$ if $n$ is even.  We therefore only need to sum over the even positive integers $n = 2k$
\begin{align*}
&= \frac{1}{2}\sum_{k=0}^\infty \frac{2}{(2k)!}2^{2k}x^{2k} \\
&= \sum_{k=0}^\infty \frac{1}{(2k)!}2^{2k} x^{2k}
\end{align*}
}

\newproblem{taylor:sinhx2series}{%
Find the Taylor series around zero for $\sinh(x^2) = \frac{1}{2}\left(e^{x^2} - e^{-x^2}\right)$.}{%
\begin{align*}
\frac{1}{2}\left(e^{x^2} - e^{-x^2}\right) &= \frac{1}{2}\left(\sum_{n=0}^\infty \frac{1}{n!}(x^2)^n - \sum_{n=0}^\infty \frac{1}{n!}(-x^2)^n\right) \\
&=\frac{1}{2}\sum_{n=0}^\infty\left( \frac{1}{n!}(x^2)^n -\frac{1}{n!}(-1)^n(x^2)^n \right)\\
&= \frac{1}{2}\sum_{n=0}^\infty \frac{1 - (-1)^n}{n!}(x^2)^n
\end{align*}
Since $1 - (-1)^n = 0$ if $n$ is even and $2$ if $n$ is odd, we only need to sum over the odd positive integers $n = 2k+1$.
\begin{align*}
&= \frac{1}{2}\sum_{k=0}^\infty \frac{2}{(2k+1)!}(x^2)^{2k+1}\\
&=\sum_{k=0}^\infty \frac{1}{(2k+1)!}x^{2(2k+1)}
\end{align*}
}

\newproblem{taylor:exp/1-x}{%
Find the degree two Taylor polynomial around $0$ of $\frac{e^x}{1-x}$ without computing any derivatives.}{%
Recall that $e^x = 1 + x + \frac{x^2}{2} + o(x^2)$ and $\frac{1}{1-x} = 1 + x + x^2 + o(x^2)$.  Then
\begin{align*}
\frac{e^x}{1-x} &= \left(1 + x + \frac{x^2}{2} + o(x^2)\right)\left(1 + x + x^2 + o(x^2)\right) \\
&= \underbrace{1\left(1 + x + x^2 + o(x^2)\right)}_\textrm{term 1} + \underbrace{x\left(1 + x + x^2 + o(x^2)\right)}_\textrm{term 2} \\
&+ \underbrace{\frac{x^2}{2}\left(1 + x + x^2 + o(x^2)\right)}_\textrm{term 3} + \underbrace{o(x^2)\left(1 + x + x^2 + o(x^2)\right)}_\textrm{term 4} \\
&= \underbrace{1 + x + x^2 + o(x^2)}_\textrm{term 1} + \underbrace{x + x^2 + o(x^2)}_\textrm{term 2} + \underbrace{\frac{x^2}{2} + o(x^2)}_\textrm{term 3} + \underbrace{o(x^2)}_\textrm{term 4} \\
&= 1 + 2x + \frac{5}{2}x^2 + o(x^2)
\end{align*}
Therefore $T_2^0 \frac{e^x}{1-x} = 1 + 2x + \frac{5}{2}x^2$.
}

\newproblem{taylor:ex3}{%
Find the degree nine Taylor polynomial around zero for $e^{x^3}$ without computing any derivatives.}{%
We know the full Taylor series for $e^x$ is given by
\begin{align*}
e^x &= \sum_{n=0}^\infty \frac{1}{n!}x^n
\end{align*}
and therefore we know that
\begin{align*}
e^{x^3} &= \sum_{n=0}^\infty \frac{1}{n!}(x^3)^n = \sum_{n=0}^\infty \frac{1}{n!}x^{3n}.
\end{align*}
We can recover the degree nine Taylor polynomial by observing that
\begin{align*}
\sum_{n=0}^\infty \frac{1}{n!}(x^3)^n = \sum_{n=0}^\infty \frac{1}{n!}x^{3n} &= 1 + x^3 + \frac{x^6}{2} + \frac{x^9}{6} + o(x^9)
\end{align*}
from which it follows that $T_9^0 e^{x^3} = 1 + x^3 + \frac{x^6}{2} + \frac{x^9}{6}$.}

\newproblem{taylor:calcplusoh}{%
Compute the degree seven Taylor polynomial around zero for $\frac{4x^3}{\left(1-x^4\right)^2}$.  Hint: You should not differentiate this function.}
{
\begin{align*}
\frac{4x^3}{\left(1-x^4\right)^2} &= \frac{d}{dx}\frac{1}{1-x^4} \\
&= \frac{d}{dx} \left(1 + \sum_{n=1}^\infty (x^4)^n\right) \\
&= 0 + \sum_{n=1}^\infty \frac{d}{dx} x^{4n} \\
&= \sum_{n=1}^\infty (4n) x^{4n-1} \\
&= 4x^3 + 8x^7 + o(x^7)
\end{align*}
Therefore $T_7^0 \frac{4x^3}{\left(1-x^4\right)2} = 4x^3 + 8x^7.$
}

\newproblem{taylor:14expminus}{%
Find $T_{14}^0 e^{x^6} - \frac{1}{1-x^5}$.}{%
\begin{align*}
e^{x^6} &= 1 + x^6 + \frac{x^{12}}{2} + \frac{x^{18}}{3!} + o(x^{18}) \\
&= 1 + x^6 + \frac{1}{2} x^{12} + o(x^{14}) \\
\frac{1}{1-x^5} &= 1 + x^5 + x^{10} + x^{15} + o(x^{15}) \\
&= 1 + x^5 + x^{10} + o(x^{14})
\end{align*}
so that
\begin{align*}
e^{x^6} - \frac{1}{1 - x^5} &= \left(1 + x^6 + \frac{1}{2} x^{12} + o(x^{14})\right) - \left(1 + x^5 + x^{10} + o(x^{14})\right) \\
&= - x^5 + x^6 - x^{10} + \frac{1}{2}x^{12} + o(x^{14})
\end{align*}
so that $T_{14}^0 \left(e^{x^6} - \frac{1}{1-x^5}\right) =-x^5 + x^6 - x^{10} + \frac{1}{2} x^{12}$.
}

\newproblem{taylor:expplusseries}{%
Find
\begin{align*}
T_\infty^0 x\left(e^x - \frac{1}{1-x}\right)
\end{align*}}{%
\begin{align*}
T_\infty^0 x\left(e^x - \frac{1}{1-x}\right) &= T_\infty^0 x\left(T_\infty^0 e^x - T_\infty^0 \frac{1}{1-x}\right) \\
&= x \left(\sum_{n=0}^\infty \frac{1}{n!}x^n - \sum_{n=0}^\infty x^n\right) \\
&= x \left(\sum_{n=0}^\infty (\frac{1}{n!} - 1)x^n\right) \\
&= \sum_{n=0}^\infty \left(\frac{1}{n!} - 1\right)x^{n+1}
\end{align*}
noticing that the first two terms in this sum are zero, we can rewrite this as $\sum_{n=2}^\infty \left(\frac{1}{n!} - 1\right)x^{n+1}$.
}

\newproblem{taylor:seriesrational}{%
Find the Taylor series around $0$ $(T_\infty^0)$ of the function $f(x) = \frac{10x^4}{(1-x^5)^2}$}{%
\begin{align*}
T_\infty \frac{10x^4}{(1-x^5)^2} &= T_\infty 2\frac{d}{dx} \frac{1}{1-x^5} \\
&= 2 \frac{d}{dx} T_\infty \frac{1}{1-x^5} \\
&=2 \frac{d}{dx} \sum_{n=0}^\infty (x^5)^n \\
&= 2 \frac{d}{dx}\left(1 + \sum_{n=1}^\infty x^{5n} \right) \\
&= 2 \sum_{n=1}^\infty \frac{d}{dx} x^{5n} \\
&= 2 \sum_{n=1}^\infty (5n)x^{5n-1} 
\end{align*}
}
