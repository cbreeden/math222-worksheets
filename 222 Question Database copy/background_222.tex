
\newproblem{bg222:deriv1}{%
Define $f(x) = x\ln(x)  + \sin(x^2+1)$.  Compute $f'(x)$.}{%
\begin{align*}
f'(x) = \ln(x) + 1 + \cos\left(x^2 + 1\right)(2x)
\end{align*}}

\newproblem{bg222:deriv2}{%
Define $f(x) = \ln(x)\sin(x) + \sqrt{x^4+x^2}$. Compute $f'(x)$.}{%
\begin{align*}
f'(x) &= \frac{1}{x}\sin(x) + \ln(x)\cos(x) + \frac{1}{2\sqrt{x^4+x^2}}\left(4x^3 + 2x \right) \\
&= \frac{1}{x}\sin(x) + \ln(x)\cos(x) + \frac{1}{2|x|\sqrt{x^2+1}}\left(4x^3 + 2x\right)
\end{align*}
}


\newproblem{bg222:deriv3}{
Compute each of the following derivatives.
\begin{enumerate}
		\item  Compute $y'$ if $y=x^3\left(x^2-\frac{1}{x}\right)$
		\item  $\frac{d}{dx} \sin(x^2+5)$
		\item  $\frac{d}{dx} \left( x^2  \cdot \cos x\right)$
		\item  $\frac{d}{dx} \left( x^2  \cdot \ln x \cdot \cos x\right)$
\end{enumerate}
}{
\begin{enumerate}
\item \begin{align*} 
y = x^3(x^2 - \frac{1}{x}) &= x^5 - x^2 \\
y' &= 5x^4 - 2x \end{align*}
\item $\frac{d}{dx} \sin(x^2 + 5) = 2x\cos(x^2 + 5)$
\item $\frac{d}{dx} x^2 \cos(x) = 2x \cos(x) - x^2\sin(x)$
\item 
\begin{align*}\frac{d}{dx} x^2 \ln(x)\cos(x) &= 2x \ln(x)\cos(x) + x^2(\frac{1}{x})\cos(x) - x^2 \ln(x)\sin(x) \\
&= 2x \ln(x) \cos(x) + x\cos(x) - x^2 \ln(x) \sin(x)
\end{align*}
\end{enumerate}
}


\newproblem{bg222:ftoc1}{% 
Define $f(x) = \int_x^{x^2} e^{t^3} dt$.  Compute $f'(x)$.  Hint: split the integral into $\int_x^1$ and $\int_1^{x^2}$ and use the Fundamental Theorem of Calculus.}
{%
\begin{align*}
f(x) &= \int_1^{x^2} e^{t^3} dt + \int_x^1 e^{t^3} dt \\
&= \int_1^{x^2} e^{t^3} dt - \int_1^x e^{t^3} dt \\ 
f'(x) &= e^{\left(x^2\right)^3}(2x) - e^{x^3}
\end{align*}
}

\newproblem{bg222:ftoc2}{%
Compute $\frac{d}{dx} \int_{x}^1 \ln z\;dz $.  (Hint:  remember the ``Fundamental Theorem of Calculus''.)\vspace{4cm}}{%
The key point is that you don't actually need to compute an antideriviatve of $\ln z$.  Namely, just imagine that we have a function $F(z)$ which
is an antiderivative of $\ln z$, i.e. a function $F(z)$ where $F'(z)=\ln z$.  Then

\begin{align*}
\frac{d}{dx} \int_{x}^1 \ln z\;dz  &= \frac{d}{dx} \left[ F(z) \right]_{z=x}^{z=1}\\
&=\frac{d}{dx}\left(  F(1)-F(x) \right) \\
&= 0 - F'(x) \\
&= -\ln x.
\end{align*}}


\newproblem{bg222:int1}{%
Compute \begin{align*}\int \frac{\ln(\pi x)}{x}dx \end{align*}}
{%
\begin{align*}
\int \frac{\ln(\pi x)}{x} dx &= \int u du & u = \ln(\pi x) \hspace{1pc} du = \frac{1}{x} dx \\
&= \frac{u^2}{2} + C \\
&= \frac{\ln^2(\pi x)}{2} + C
\end{align*}}


\newproblem{bg222:int2}{% 
Compute $\int_e^{e^3} \frac{1}{x\ln(x)} dx$.}
{%
\begin{align*}
\int_e^{e^3} \frac{1}{x\ln(x)} &= \int_1^3 \frac{1}{u}du & u = \ln(x) \hspace{1pc} du = \frac{1}{x}dx \\
&= \ln(u)|_1^3 = \ln(3) - \ln(1) \\
&= \ln(3)\end{align*}}

\newproblem{bg222:int3}{%
Compute $\int_0^{\frac{\pi}{2}} \cos(t) e^{\sin(t)} dt$.}{%
\begin{align*}
\int_0^{\frac{\pi}{2}} \cos(t) e^{\sin(t)} dt &= \int_0^1 e^u du & u = \sin(t) \hspace{1pc} du = \cos(t)dt \\
&= e^u|_0^1 \\ 
&= e - 1
\end{align*}}

\newproblem{bg222:int4}{%
Compute $\int e^x \sin(2\pi e^x) dx$}{%
\begin{align*}
\int e^x \sin(2\pi e^x) dx &= \int \frac{1}{2\pi} \int \sin(u) du & u = 2\pi e^{x} \hspace{1pc} \frac{1}{2\pi} du =  e^x dx \\
&= \frac{-1}{2\pi} \cos(u) + C \\
&= \frac{-1}{2\pi} \cos(2\pi e^x) + C
\end{align*}
}

\newproblem{bg222:int5}{%
Compute $\int_0^x \left(\int_0^t \cos(s) ds\right) dt$.}{%
We start off by computing $\int_0^t \cos(s) ds = \sin(s)|_0^t = \sin(t) - \sin(0) = \sin(t)$.  Then 
\begin{align*}
\int_0^x \left(\int_0^t \cos(s) ds\right) dt &= \int_0^x \sin(t) dt \\
&= -\cos(t)|_0^x = -\cos(x) - (-\cos(0)) \\
&= 1 - \cos(x)
\end{align*}
}

\newproblem{bg222:int6}{
Compute each of the following integrals.
\begin{enumerate}
		\item  $\int x^2\sin(x^3)dx$
		\item $\int \frac{x^3+\sqrt{x}+\sqrt{\pi}}{\sqrt{x}}dx$
		\item  $\int_{-2}^3 (x^2+x+1)dx$
\end{enumerate}
}{
\begin{enumerate}
\item \begin{align*}
\int x^2\sin(x^3)dx &= \frac{1}{3} \int \sin(u) du & u = x^3 \hspace{1pc} \frac{1}{3}du = x^2 dx \\
&= -\frac{1}{3}\cos(u) + C \\
&= -\frac{1}{3}\cos(x^3) + C
\end{align*}
\item
\begin{align*}
\int \frac{x^3+\sqrt{x}+\sqrt{\pi}}{\sqrt{x}}dx &= \int x^{\frac{5}{2}}+1+\sqrt{\pi}\frac{1}{\sqrt{x}}dx \\
&= \frac{2}{7}x^{\frac{7}{2}} + x + 2\sqrt{\pi} \sqrt{x} + C
\end{align*}
\end{enumerate}
}

\newproblem{bg222:tf1}{%
$\frac{d}{dx} (\frac{1}{x}) = \ln x.$}{%
}

\newproblem{background:TF}{%
True or False:
\begin{enumerate}
	\item $\frac{d}{dx} (\frac{1}{x}) = \ln x$
	\item $\frac{d}{dt} \int_{0}^t \frac{dx}{1+x^2} = \frac{1}{1+t^2}$
	\item  $\sqrt{x^2+9}=x+3$
	\item  The function $f(x)=\frac{1}{x+4}$ is defined for all values of $x$ except for $x=-4$
	\item  $\int e^{(x^3)} = e^{(x^3)}+C$
	\item  If $f(x)=x^2\cdot g(x)$ then $f'(x) = 2x\cdot g'(x)$
\end{enumerate}
}{%
\begin{enumerate}
	\item False.  \begin{align*}  \frac{d}{dx} (\frac{1}{x})&= \frac{d}{dx} x^{-1} = -x^{-2}\end{align*}
	\item True.
	\item  False.
	\item  True.
	\item  False.  You $\frac{d}{dx} e^{(x^3)}+C=3x^2e^{(x^3)}$ by the Chain Rule.
	\item  False.  By the product rule, we have $f'(x)=2x\cdot g(x) + x^2g'(x)$.
\end{enumerate}
}


\newproblem{bg222:optimization}{
Which rectangle has the largest area, among all those rectangles for which the total length of the sides is $1$?}{
If the sides of the rectangle have lengths $x$ and $y$, then the total length of
the sides is 
\begin{align*}
L &= x+x+y+y = 2(x+y)
\end{align*}
and the area of the rectangle is 
\begin{align*}
A = xy.
\end{align*}
So are asked to find the largest possible value of $A=xy$ provided $2(x+y)=1$.
The lengths of the sides can also not be negative, so $x$ and $y$ must satisfy
$x\geq0$, $y\geq0$.

We now want to turn this problem into a question of the form ``maximize a function over some interval.''  The quantity which we are asked to maximize is $A$, but it depends on two variables $x$ and $y$ instead of just one variable.  However, the variables $x$ and
$y$ are not independent since we are only allowed to consider rectangles with $L=1$.  From this equation we get
\begin{align*}
L=1\implies y = \tfrac12-x.
\end{align*}
Hence we must find the maximum of the quantity
\begin{align*}
A= xy = x\bigl(\tfrac12-x\bigr)
\end{align*}
The values of $x$ which we are allowed to consider are only limited by the
requirements $x\geq 0$ and $y\geq0$, i.e.\ $x\leq \tfrac12$.  So we end up with
this problem:
\begin{center}
\textit{Find the maximum of the function $f(x)=x\bigl(\tfrac12-x\bigr)$ on the interval $0\leq x\leq \tfrac12$.}
\end{center}
Before we start computing anything we note that the function $f$ is a polynomial so that it is differentiable, and hence continuous, and also that the interval $0\leq x\leq\tfrac12$ is closed.  Therefore the theory guarantees that there is a maximum and our recipe will show us where it is.

The derivative is given by
\begin{align*}
f'(x) = \tfrac12 - 2x,
\end{align*}
and hence the only stationary point is \( x=\tfrac14 \).  The function value at this point is
\begin{align*}
f\tfrac14) = \tfrac14(\tfrac12-\tfrac14) = \tfrac1{16}.
\end{align*}
At the endpoints one has $x=0$ or $x=\tfrac12$, which corresponds to a
rectangle one of whose sides has length zero.  The area of such
rectangles is zero, and so this is not the maximal value we are
looking for.  

We conclude that the largest area is attained by the rectangle whose sides have
lengths
\begin{align*}
x=\tfrac14, \text{ and }y= \tfrac12-\tfrac14 = \tfrac14,
\end{align*}
i.e.\ by a square with sides $\tfrac14$.}





