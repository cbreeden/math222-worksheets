\newproblem{vectorprob:nonsense}{%
Let $\vec{\mathbf{a}} = \vtor[2,1,5]$, $\vec{\mathbf{b}} = \vtor[-1,2,-1]$ and $\vec{\mathbf{c}} = \vetor[2,4]$.  Which of the following expressions are nonsense?  Evaluate the sensible ones.
\begin{enumerate}
\item $3\vec{\mathbf{a}} + \vec{\mathbf{b}}$
\item $\vec{\mathbf{a}} + \vec{\mathbf{c}}$
\item $\vec{\mathbf{a}}\cdot\vec{\mathbf{c}}$
\item $\vec{\mathbf{a}} - 2\vec{\mathbf{b}}$
\item $t \vec{\mathbf{a}}$ where $t$ is a real number.
\item $\vec{\mathbf{a}} \hspace{.1pc} \vec{\mathbf{b}}$
\item $\vec{\mathbf{a}} + 5$
\end{enumerate}}{%
\begin{enumerate}
\item $3\vec{\mathbf{a}} + \vec{\mathbf{b}} = \vtor[5,5,14]$
\item nonsense
\item nonsense
\item $\vec{\mathbf{a}} - 2\vec{\mathbf{b}} = \vtor[4,-3,7]$
\item $t \vec{\mathbf{a}} = \vtor[2t,t,5t]$
\item nonsense
\item nonsense
\end{enumerate}
}

\newproblem{vectorprob:basislinear}{%
Let $\vec{\mathbf{a}} = \vetor[2,4]$ and $\vec{\mathbf{b}} = \vetor[-1,1]$. Find $s$ and $t$ so that $\vetor[3,5] = s\vec{\mathbf{a}} + t \vec{\mathbf{b}}$.}{%
We can rewrite this problem as
\begin{align*}
\vetor[2s - t,4s + t] &= \vetor[3,5]
\end{align*}
We can solve the equation $2s - t = 3$ for $t$ to find that $t = 2s - 3$.  Plugging this in to the equation $4s + t = 5$ we find that $4s + 2s - 3 = 5$ so $6s = 8$ and $s = \frac{4}{3}$.  Plugging this back in, we find that $t = 2(\frac{4}{3}) - 3 = \frac{-1}{3}$.
}

\newproblem{vectorprob:orbits}{%
Planet X rotates around a super-massive black hole, and a satellite rotates around planet $X$.  The black hole is a lot more massive than planet X, which is a lot more massive than the satellite, so their orbits are nearly circular.  Planet X orbits the black hole at a constant radius of $10^9$ miles and goes around twice in a year.  The satellite orbits planet X in the same plane, at a constant radius of $10^3$ miles, and goes around 100 times in a year.  Taking the center of the black hole as the origin, parametrize the position $\vec{\mathbf{p}}(t)$ of the satellite.}{%
Let $\vec{\mathbf{p}}_X(t)$ be the vector from the black hole to planet X and $\vec{\mathbf{p}}_S$ be the vector from (the center of) planet X to the satellite.  Then 
\begin{align*}
\vec{\mathbf{p}} &= \vec{\mathbf{p}}_X + \vec{\mathbf{p}}_S
\end{align*}

Depending on the coordinates you choose, your expression for $p_X$ and $p_S$ may be slightly different than mine.  Choosing coordinates in one natural way, the position of a point moving around a circle of radius $R$ at a rate of $n$ revolutions per year is 
\begin{align*}
\vetor[R\cos (2\pi n), R\sin (2\pi n)]
\end{align*}

Choose coordinates so that the black hole, the planet, and the satellite are lined up in that order pointing in the $x$ direction at time $t=0$.  Then 
\begin{align*}
\vec{\mathbf{p}}_X(t) = \vetor[10^9\cos (2\pi n), 10^9\sin (2\pi n)]
\end{align*}

\begin{align*}
\vec{\mathbf{p}}_S(t) = \vetor[10^3 \cos \left(2\pi \frac{t}{100}\right), 10^3 \sin \left(2\pi \frac{t}{100}\right)]
\end{align*}

Thus 
\begin{align*}
\vec{\mathbf{p}}(t) = \vetor[10^9 \cos \left(2\pi \frac{t}{2} \right) + 10^3 \cos\left(2\pi \frac{t}{100} \right), 10^9 \sin\left(2\pi \frac{t}{2} \right) + 10^3 \sin\left(2\pi \frac{t}{100} \right)]
\end{align*}
}

\newproblem{vectorprob:projquiz1}{%
Let $\vec{\mathbf{a}} = \vtor[1,2,-2]$ and $\vec{\mathbf{b}} = \vtor[1,0,1]$.  Find $\vec{\mathbf{a}}^{\pll}$ and $\vec{\mathbf{a}}^{\perp}$ so that $\vec{\mathbf{a}} = \vec{\mathbf{a}}^{\pll} + \vec{\mathbf{a}}^{\perp}$, where $\vec{\mathbf{a}}^{\pll}$ is parallel to $\vec{\mathbf{b}}$ and $\vec{\mathbf{a}}^{\perp}$ is perpendicular to $\vec{\mathbf{b}}$.}{%
\begin{align*}
\vec{\mathbf{a}}^{\pll} &= \left(\vec{\mathbf{a}} \cdot \frac{\vec{\mathbf{b}}}{\|\vec{\mathbf{b}}\|}\right)\frac{\vec{\mathbf{b}}}{\|\vec{\mathbf{b}}\|} \\
&= \left(\vec{\mathbf{a}}\cdot \vec{\mathbf{b}}\right) \frac{\vec{\mathbf{b}}}{\|\vec{\mathbf{b}}\|^2}
\end{align*}
 Observe that $\|\vec{\mathbf{b}}\| = \sqrt{1 + 1} = \sqrt{2}$ and $\left(\vec{\mathbf{a}} \cdot \vec{\mathbf{b}}\right) = 1 - 2 = -1$.  Then
 \begin{align*}
 \vec{\mathbf{a}}^{\pll} &= \vtor[-\frac{1}{2},0,-\frac{1}{2}]
 \end{align*} 
 
 and
\begin{align*}
\vec{\mathbf{a}}^{\perp} &= \vec{\mathbf{a}} - \vec{\mathbf{a}}^{\pll} \\
&= \vtor[1,2,-2] - \vtor[-\frac{1}{2},0,-\frac{1}{2}] \\
&= \vtor[\frac{3}{2},2,-\frac{3}{2}]
\end{align*}
}

\newproblem{vectorprob:projquiz2}{%
Let $\vec{\mathbf{a}} = \vtor[-1,2,2]$ and $\vec{\mathbf{b}} = \vtor[1,0,1]$.  Find $\vec{\mathbf{a}}^{\pll}$ and $\vec{\mathbf{a}}^{\perp}$ so that $\vec{\mathbf{a}} = \vec{\mathbf{a}}^{\pll} + \vec{\mathbf{a}}^{\perp}$, where $\vec{\mathbf{a}}^{\pll}$ is parallel to $\vec{\mathbf{b}}$ and $\vec{\mathbf{a}}^{\perp}$ is perpendicular to $\vec{\mathbf{b}}$.}{%
\begin{align*}
\vec{\mathbf{a}}^{\pll} &= \left(\vec{\mathbf{a}} \cdot \frac{\vec{\mathbf{b}}}{\|\vec{\mathbf{b}}\|}\right)\frac{\vec{\mathbf{b}}}{\|\vec{\mathbf{b}}\|} \\
&= \left(\vec{\mathbf{a}}\cdot \vec{\mathbf{b}}\right) \frac{\vec{\mathbf{b}}}{\|\vec{\mathbf{b}}\|^2}
\end{align*}
 Observe that $\|\vec{\mathbf{b}}\| = \sqrt{1 + 1} = \sqrt{2}$ and $\left(\vec{\mathbf{a}} \cdot \vec{\mathbf{b}}\right) = -1 + 2 = 1$.  Then
 \begin{align*}
 \vec{\mathbf{a}}^{\pll} &= \vtor[\frac{1}{2},0,\frac{1}{2}]
 \end{align*} 
 
 and
\begin{align*}
\vec{\mathbf{a}}^{\perp} &= \vec{\mathbf{a}} - \vec{\mathbf{a}}^{\pll} \\
&= \vtor[-1,2,2] - \vtor[\frac{1}{2},0,\frac{1}{2}] \\
&= \vtor[-\frac{3}{2},2,\frac{3}{2}]
\end{align*}
}

\newproblem{vectorprob:proj1}{%
Let $\vec{\mathbf{a}} = \vtor[2,1,1]$ and $\vec{\mathbf{b}} = \vtor[1,1,0]$.  Find $\vec{\mathbf{a}}^{\pll}$ and $\vec{\mathbf{a}}^{\perp}$ so that $\vec{\mathbf{a}} = \vec{\mathbf{a}}^{\pll} + \vec{\mathbf{a}}^{\perp}$, where $\vec{\mathbf{a}}^{\pll}$ is parallel to $\vec{\mathbf{b}}$ and $\vec{\mathbf{a}}^{\perp}$ is perpendicular to $\vec{\mathbf{b}}$.}{%
\begin{align*}
\vec{\mathbf{a}}^{\pll} &= \left(\vec{\mathbf{a}} \cdot \frac{\vec{\mathbf{b}}}{\|\vec{\mathbf{b}}\|}\right)\frac{\vec{\mathbf{b}}}{\|\vec{\mathbf{b}}\|} \\
&= \left(\vec{\mathbf{a}}\cdot \vec{\mathbf{b}}\right) \frac{\vec{\mathbf{b}}}{\|\vec{\mathbf{b}}\|^2}
\end{align*}
 Observe that $\|\vec{\mathbf{b}}\| = \sqrt{1 + 1} = \sqrt{2}$ and $\left(\vec{\mathbf{a}} \cdot \vec{\mathbf{b}}\right) = 2 + 1 = 3$.  Then
 \begin{align*}
 \vec{\mathbf{a}}^{\pll} &= \vtor[\frac{3}{2},\frac{3}{2},0]
 \end{align*} 
 
 and
\begin{align*}
\vec{\mathbf{a}}^{\perp} &= \vec{\mathbf{a}} - \vec{\mathbf{a}}^{\pll} \\
&= \vtor[2,1,1] - \vtor[\frac{3}{2},\frac{3}{2},0] \\
&= \vtor[\frac{1}{2},-\frac{1}{2},1]
\end{align*}
}

\newproblem{vectorprob:param1}{%
Find a parametric equation for the line that passes through the points $A = (1, 0, 2)$ and $B = (3, 1, 4)$.}{%
\begin{align*}
\vec{AB} &= \vtor[3 - 1, 1 - 0, 4 - 2] \\
&= \vtor[2,1,2]
\end{align*}
so a parametric equation for the line is
\begin{align*}
\vec{\mathbf{l}}(t) &= \vtor[1,0,2] + \vtor[2,1,2]t
\end{align*}
}

\newproblem{vectorprob:word1}{%
Suppose that a merchant sells three types of goods in quantities $q_1$, $q_2$, and $q_3$ and that the merchant sells these goods at prices $p_1$, $p_2$, and $p_3$ dollars per unit respectively.  Suppose further that it costs the merchant $c_i$ dollars to make one unit of the $i^{th}$ good.  If
\begin{align*}
\vec{\mathbf{q}} &= \vtor[q_1, q_2, q_3] & \vec{\mathbf{p}} &= \vtor[p_1, p_2, p_3] & \vec{\mathbf{c}} &= \vtor[c_1, c_2, c_3]
\end{align*}
then what is the significance of the quantity
\begin{align*}
\vec{\mathbf{q}}\cdot\left(\vec{\mathbf{p}} - \vec{\mathbf{c}}\right)?
\end{align*}
Describe in words why we the merchant cares if this quantity is positive or negative.
}{%
$\hspace{.1pc} \vec{\mathbf{q}}\cdot\left(\vec{\mathbf{p}} - \vec{\mathbf{c}}\right)$ is the profit that the merchant makes by producing $q_1$ units of the first good, $q_2$ units of the second good, and $q_3$ units of the third good.  The merchant cares because he or she wants to make money.
}

\newproblem{vectorprob:plane1}{%
Does the plane containing the points $A = (1,0,0)$, $B = (0,1,0)$, and $C =  (0,0,1)$ also contain the point $(1,1,1)$?}{%
We would like to find the normal vector to the plane so that we can write the equation for the plane in standard form.  If we do that, we can check whether or not $(1,1,1)$ satisfies the equation for this plane.  To find the normal, we need two vectors in the plane.
\begin{align*}
\overset{\longrightarrow}{AB} &= \vtor[-1,1,0] \\
\overset{\longrightarrow}{BC} &= \vtor[0,-1,1] \\
\overset{\longrightarrow}{AB} \times \overset{\longrightarrow}{BC} &= 
\left|\begin{array}{ccc}
\vec{\mathbf{i}} & -1 & 0 \\
\vec{\mathbf{j}} & 1 & -1 \\
\vec{\mathbf{k}} & 0 &  1\\
\end{array}\right|
\end{align*}
We can compute that $\overset{\longrightarrow}{AB} \times \overset{\longrightarrow}{BC} = \vec{\mathbf{i}} + \vec{\mathbf{j}} + \vec{\mathbf{k}} = \vtor[1,1,1]$.  From this we can see that the equation for the plane is $x + y + z = 1$ and the point $(1,1,1)$ does not satisfy this equation.}