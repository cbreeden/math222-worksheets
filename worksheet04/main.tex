%%%%%%%%%%%%%%%%%%%%%%%%%%%%%%%%%%%%%%%%%
% Short Sectioned Assignment
% LaTeX Template
% Version 1.0 (5/5/12)
%
% This template has been downloaded from:
% http://www.LaTeXTemplates.com
%
% Original author:
% Frits Wenneker (http://www.howtotex.com)
%
% License:
% CC BY-NC-SA 3.0 (http://creativecommons.org/licenses/by-nc-sa/3.0/)
%
%%%%%%%%%%%%%%%%%%%%%%%%%%%%%%%%%%%%%%%%%

%----------------------------------------------------------------------------------------
%	PACKAGES AND OTHER DOCUMENT CONFIGURATIONS
%----------------------------------------------------------------------------------------

\documentclass[paper=a4, fontsize=11pt]{scrartcl} % A4 paper and 11pt font size
\usepackage{geometry} 
\geometry{left=0.8in, right =0.8in, bottom=1in, top=0.6in} % Establish a 1 inch margin

\usepackage{graphicx} % Allow ability to include graphics

\usepackage[T1]{fontenc} % Use 8-bit encoding that has 256 glyphs
\usepackage{fourier} % Use the Adobe Utopia font for the document
\usepackage[english]{babel} % English language/hyphenation
\usepackage{amsmath,amsfonts,amsthm} % Math packages

\usepackage{enumitem}
\usepackage{tabularx}

\usepackage{sectsty} % Allows customizing section commands
\allsectionsfont{\centering \normalfont\scshape\underline} % Make all sections centered, the default font and small caps

\usepackage{fancyhdr} % Custom headers and footers
\pagestyle{fancyplain} % Makes all pages in the document conform to the custom headers and footers
\fancyhead{} % No page header - if you want one, create it in the same way as the footers below
\fancyfoot[L]{} % Empty left footer
\fancyfoot[C]{} % Empty center footer
\fancyfoot[R]{\thepage} % Page numbering for right footer
\renewcommand{\headrulewidth}{0pt} % Remove header underlines
\renewcommand{\footrulewidth}{0pt} % Remove footer underlines
\setlength{\headheight}{13.6pt} % Customize the height of the header

\numberwithin{equation}{section} % Number equations within sections (i.e. 1.1, 1.2, 2.1, 2.2)
\numberwithin{figure}{section} % Number figures within sections (i.e. 1.1, 1.2, 2.1, 2.2)
\numberwithin{table}{section} % Number tables within sections (i.e. 1.1, 1.2, 2.1, 2.2)

\setlength\parindent{0pt} % Removes all indentation from paragraphs - comment this line for an assignment with lots of text

%----------------------------------------------------------------------------------------
%	TITLE SECTION
%----------------------------------------------------------------------------------------
\title{Math 222 - Worksheet 4}
\newcommand{\horrule}[1]{\rule{\linewidth}{#1}} % Create horizontal rule command with 1 argument of height

\begin{document}

%% Header
{
\normalfont \normalsize 
\begin{flushright}
\textsc{Math 222, UW Madison}
\end{flushright}
{\center
\horrule{0.5pt} \\[0.4cm] % Thin top horizontal rule
{\huge Worksheet 4}\\
Reduction Formulas \\ % The assignment title
\horrule{2pt} \\[0.5cm] % Thick bottom horizontal rule
}}


%----------------------------------------------------------------------------------------
%	PROBLEM 1
%----------------------------------------------------------------------------------------

You often find that you just need to apply integration by parts a lot; I really mean a lot, like if you want to solve $\displaystyle \int \sin^{64}(x)\,dx$.  So to simplify doing the same thing over and over again, we write down a formula for the reduction of the integral.  These are called {\bfseries reduction formulas}; computers love them.

\section*{Using Reduction Formulas}

1. Using the reduction formula $\displaystyle \int x^n e^x\,dx = x^n e^x - n\int x^{n-1}e^x\,dx$, compute $\displaystyle \int x^4 e^x\,dx$.

\vfill

2. Compute $\displaystyle \int_0^{\pi} \cos(x)\,dx$ and $\displaystyle \int_0^{\pi} x^2 \cos(x)\,dx$.  Then use the reduction formula:

\[ \int x^n\cos(x)\,dx = x^n\sin(x) + nx^{n-1}\cos(x) - n(n-1)\int x^{n-2}\cos(x)\,dx \]

to compute $\displaystyle \int_0^{\pi} x^4 \cos(x)\,dx$.

\vfill
\vfill

\newpage

3. Use the reduction formula $\displaystyle \int \tan^n(x)\,dx = \frac{\tan^{n-1}(x)}{n-1} - \int \tan^{n-2}(x)\,dx$ to calculate $\displaystyle \int_0^{\pi/4} \tan^5(x)\,dx$.

\vfill


\section*{Deriving Reduction Formulas}

4. Show $\displaystyle \int x^n e^x\,dx = x^n e^x - n \int x^{n-1} e^x\,dx$\\
\vfill

5. Show $\displaystyle \int \tan^n(x)\,dx = \frac{\tan^{n-1}(x)}{n-1} - \int \tan^{n-2}(x)\,dx$.\\
(most of the time, deriving a reduction formula comes down to IBP, but not always).
\vfill







\end{document}




