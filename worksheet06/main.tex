%%%%%%%%%%%%%%%%%%%%%%%%%%%%%%%%%%%%%%%%%
% Short Sectioned Assignment
% LaTeX Template
% Version 1.0 (5/5/12)
%
% This template has been downloaded from:
% http://www.LaTeXTemplates.com
%
% Original author:
% Frits Wenneker (http://www.howtotex.com)
%
% License:
% CC BY-NC-SA 3.0 (http://creativecommons.org/licenses/by-nc-sa/3.0/)
%
%%%%%%%%%%%%%%%%%%%%%%%%%%%%%%%%%%%%%%%%%

%----------------------------------------------------------------------------------------
%	PACKAGES AND OTHER DOCUMENT CONFIGURATIONS
%----------------------------------------------------------------------------------------

\documentclass[letterpaper, fontsize=11pt]{scrartcl} % A4 paper and 11pt font size
\usepackage{geometry} 
\geometry{left=0.8in, right =0.8in, bottom=1in, top=0.6in} % Establish a 1 inch margin

\usepackage{graphicx} % Allow ability to include graphics

\usepackage[T1]{fontenc} % Use 8-bit encoding that has 256 glyphs
\usepackage{fourier} % Use the Adobe Utopia font for the document
\usepackage[english]{babel} % English language/hyphenation
\usepackage{amsmath,amsfonts,amsthm} % Math packages

\usepackage{enumitem}
\usepackage{tabu}

\usepackage{sectsty} % Allows customizing section commands
\allsectionsfont{\centering \normalfont\scshape\underline} % Make all sections centered, the default font and small caps

\usepackage{fancyhdr} % Custom headers and footers
\pagestyle{fancyplain} % Makes all pages in the document conform to the custom headers and footers
\fancyhead{} % No page header - if you want one, create it in the same way as the footers below
\fancyfoot[L]{} % Empty left footer
\fancyfoot[C]{} % Empty center footer
\fancyfoot[R]{\thepage} % Page numbering for right footer
\renewcommand{\headrulewidth}{0pt} % Remove header underlines
\renewcommand{\footrulewidth}{0pt} % Remove footer underlines
\setlength{\headheight}{13.6pt} % Customize the height of the header

\numberwithin{equation}{section} % Number equations within sections (i.e. 1.1, 1.2, 2.1, 2.2)
\numberwithin{figure}{section} % Number figures within sections (i.e. 1.1, 1.2, 2.1, 2.2)
\numberwithin{table}{section} % Number tables within sections (i.e. 1.1, 1.2, 2.1, 2.2)

\setlength\parindent{0pt} % Removes all indentation from paragraphs - comment this line for an assignment with lots of text

%----------------------------------------------------------------------------------------
%	TITLE SECTION
%----------------------------------------------------------------------------------------
\title{Math 222 - Worksheet 6}
\newcommand{\horrule}[1]{\rule{\linewidth}{#1}} % Create horizontal rule command with 1 argument of height

\begin{document}

%% Header
{
\normalfont \normalsize 
\begin{flushright}
\textsc{Math 222, UW Madison}
\end{flushright}
{\center
\horrule{0.5pt} \\[0.4cm] % Thin top horizontal rule
{\huge Worksheet 6}\\
Trig \& UV Substitution \\ % The assignment title
\horrule{2pt} \\[0.5cm] % Thick bottom horizontal rule
}}


%----------------------------------------------------------------------------------------
%	PROBLEM 1
%----------------------------------------------------------------------------------------

\vfill

\section*{Useful Formulas for Trig Sub}

\begin{center}\tabulinesep=5pt
\begin{tabu} to 0.8\linewidth { X[1,$$c] X[1,c] X[1,c] }
  \sin^2 \theta = 1 - \cos^2 \theta & for & $\sqrt{a^2 - x^2}$ \\
  1 + \tan^2 \theta = \sec^2 \theta & for & $\sqrt{x^2 + a^2}$ \\
  \sec^2 \theta - 1 = \tan^2 \theta & for & $\sqrt{x^2 - a^2}$ \\

  \textsc{Soh Cah Toa} & for & $\sin(\arccos x)$\newline%
  							   $\cos(\arcsin x)$\newline%
                               etc\dots\\

  \sin(2\theta) = 2\sin \theta \cos \theta  & for & $\sin(2 \arccos x)$\newline%
  													$\sin(2 \arctan x)$\newline%
                                                    etc\dots\\
                                                    
  \text{Completing the Square} & for & anything math related

\end{tabu}
\end{center}

\vfill\vfill

\section*{Useful Formulas for UV Sub}

\begin{center}\tabulinesep=5pt
\begin{tabu} to 0.8\linewidth { X[1,$$c] X[1,c] X[1,c] }
  \multicolumn{3}{c}{$\displaystyle U(t) = \frac{1}{2} \Big(t + \frac{1}{t}\Big) \qquad%
                     V(t) = \frac{1}{2} \Big(t - \frac{1}{t}\Big)$} \\
  
  \multicolumn{3}{c}{$\displaystyle t = U(t) + V(t) \qquad \frac{1}{t} = U(t) - V(t)$} \\
  1 + V^2(t) = U^2(t) & for & $\sqrt{x^2 + a^2}$ \\
  U^2(t) - 1 = V^2(t) & for & $\sqrt{x^2 - a^2}$ \\
\end{tabu}
\end{center}

\vfill\vfill

\newpage

\section*{Using Trig-Sub}

1. \quad $\displaystyle \int \sqrt{16 - x^2}\,dx$.

\vfill

2. \quad $\displaystyle \int \frac{x^3}{\sqrt{4 - x^2}}\,dx$

\vfill

3. \quad $\displaystyle \int \frac{e^t}{\sqrt{4 - e^{2t}}}\,dx$

\vfill

4. \quad $\displaystyle \int \frac{1}{x^2 + 2x + 5}\,dx$

\vfill

\newpage

\section*{Using UV-Sub}

Concerning UV-Sub, I usually recommend trying an appropriate trig substitution first and if you get a trig integral that you aren't sure how to solve then trying UV-sub.\\

5. \quad $\displaystyle \int \sqrt{x^2 - 4} \,dx$

\vfill

6. \quad $\displaystyle \int \sqrt{4 + x^2} \,dx$

\vfill

\end{document}




